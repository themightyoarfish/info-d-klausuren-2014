\documentclass{scrartcl}
\title{Info D Klausur SS2014 Gruppe Reis}
\author{Rasmus Diederichsen}
\date{\today}
\usepackage{microtype,
   textcomp,
   lastpage,
   titlesec,
   listings,
   xcolor,
   IEEEtrantools,
   tabularx,
   tikz,
   fancyhdr,
   array,
   amsmath,
   amssymb,
   graphicx,
   subcaption,
lmodern}
\usepackage[ngerman]{babel}
\usepackage[top=2cm,bottom=2.5cm,right=2cm,left=2cm]{geometry}
\usepackage[pdftitle={Info D Klausur SS2014 Gruppe Reis}, 
   pdfauthor={Rasmus Diederichsen}, 
   hyperfootnotes=true,
   colorlinks,
   bookmarksnumbered = true,
   linkcolor = black,
   plainpages = false,
citecolor = lightgray]{hyperref}
\lstset{
   basicstyle=\footnotesize\ttfamily,
   language=Python,
   numbersep=-5pt,
   breaklines=true,
   commentstyle=\color{blue},
   keywordstyle=\color{purple}\textbf,
   numberstyle=\tiny\color{gray},
   numbers=left,
   stringstyle=\color{olive},
}
\usepackage[utf8]{inputenc}
\usepackage[T1]{fontenc}
\usepackage[all]{hypcap}
\titleformat{\subsection}{\normalfont\fontsize{12}{15}\bfseries}{\thesubsection}{1em}{}
\titleformat{\section}{\normalfont\fontsize{12}{15}\bfseries}{\thesection}{1em}{}
\DeclareMathOperator*{\op}{\bigcirc}
\renewcommand{\headrulewidth}{0pt}
\renewcommand{\footrulewidth}{0pt}
\pagestyle{fancy}
\fancyhf{}
\fancyfoot[L]{\emph{Blatt \thepage{} von \pageref{LastPage}}}
\begin{document}

\section*{Aufgabe 1: Sprache vs. Grammatik \hfill (12 Punkte)} 

\subsection*{(a)} 

\subsection*{(b) Grammatikdefinition \hfill{} \normalfont{(2 Punkte)}} 
Definieren Sie kontextfreie Grammatiken (für Sprachen $L$ mit
\vspace{\baselineskip}
$\varepsilon\not\in L$).

\begin{center}
   \noindent\fcolorbox{lightgray}{lightgray!20}{
      \begin{minipage}{.95\textwidth}
         \emph{Alle Regeln haben die Form\ldots}
         \vspace{\baselineskip}
         \begin{equation*}
            V \rightarrow (V \cup \Sigma)^+
         \end{equation*}
         \vspace{\baselineskip}
      \end{minipage}
   }
\end{center}

\section*{Aufgabe 2: Sprachen \hfill{} (12 Punkte)}

\newcommand{\mpsol}{\colorbox{lightgray!20}{$\boxtimes$}}
\renewcommand{\mp}{\colorbox{lightgray!20}{$\square$}}

\renewcommand{\arraystretch}{1.4}
\begin{tabularx}{\textwidth}{ccX}
   korrekt & falsch & \\ \hline
   \mpsol & \mp & Die Sprache $\left\{0^k1^\ell2^k22^\ell \mid k,\ell\ge 0\right\}$ ist kontextfrei.\\
   \mpsol & \mp & Nicht-deterministische Kellerautomaten, die maximal ein Symbol im Keller
   speichern können, sind nur so mächtig wie deterministische
   endliche Automaten. \\
   \mp & \mpsol & Man kann jeden deterministischen Kellerautomaten in einen
   nichtdeterministischen Kellerautomaten umformen, dessen Keller immer maximal 2
   Elemente enthält.\\
   \mp & \mpsol & Es gibt endliche Mengen, die nicht das Alphabet einer Sprache
   sein können \\
   \mpsol & \mp & Das Alphabet einer Sprache kann unendlich groß sein.\\
   \mp & \mpsol & $\Sigma^*$ ist die Potenzmenge von $\Sigma$.
\end{tabularx}

\pagebreak 

\section*{Aufgabe 8: Entscheidbarkeit \hfill{} (12 Punkte)}

Wir definieren das Ergebnis des \emph{Klebeoperators} $\circ$ als die Zahl, die
durch das Hintereinanderschreiben der Dezimaldarstellungen ihrer einzelnen
Argumente repräentiert wird. Wir können mehrere Klebeoperationen gesammelt
schreiben, z.B.

\begin{equation*}
   \op\limits_{i=1}^4 i^2 = 1 \circ 4 \circ 9 \circ 16 = 14916.
\end{equation*}
Betrachten Sie das folgende Problem:
\begin{center}
   \fbox{
      \begin{minipage}{.95\textwidth}
         \textbf{Gegeben}: Drei jeweils $m-$elementige Mengen $\mathcal{A} :=
         \left\{a_i\right\}_{1\le i\le k}$, $\mathcal{B} :=
         \left\{b_i\right\}_{1\le i\le k}$, $\mathcal{C} :=
         \left\{c_i\right\}_{1\le i\le k}$ mit Elementen aus $\mathbb{N}$.

         \noindent\textbf{Frage}: Gibt es eine Sequenz $s(1),s(2),\ldots,s(n)$ mit $n \ge
         1$ und $s(i) \in \{1,\ldots,k\}$ für alle $1 \le i \le n$, sodass 

         \begin{tabularx}{\textwidth}{>{\raggedleft\arraybackslash}p{.5\textwidth}cl>{\raggedleft\arraybackslash}X}
         % \begin{IEEEeqnarray*}{+rCl+r*}
            $\op\limits_{i=1}^na_{s(i)} - \op\limits_{i=1}^nb_{s(i)}$ & = & $\op\limits_{i=1}^nc_{s(i)}$ &  ("`Gleichung"')
         \end{tabularx}
      \end{minipage}
   }
\end{center}

\subsection*{(a) Beispiele \hfill{} \normalfont{(4 Punkte)}}

Geben Sie jeweils ein Beispiel einer Ja- und einer Nein-Instanz für dieses
Problem an:\\[.5\baselineskip]
Ja-Instanz \hfill{} Nein-Instanz

\vspace{2pt}
\setlength{\fboxsep}{10pt}
   \noindent\fcolorbox{lightgray}{lightgray!20}{
      \begin{minipage}[t][3cm]{.40\textwidth}
         Sei $\mathcal{A} = \mathcal{B} := \left\{1,1,1\right\}, \mathcal{C}
         := \{0,0,0\}$. Offensichtlich ist die Instanz lösbar mit $s(1) =
         s(2) = s(3) \in \{1,2,3\}$.
      \end{minipage}
   }
   \hfill{}\fcolorbox{lightgray}{lightgray!20}{
      \begin{minipage}[t][3cm]{.40\textwidth}
         $\mathcal{A} = \mathcal{B} = \mathcal{C} := \left\{1,2,3\right\}$
      \end{minipage}
   }

\subsection*{(b) Semi-Entscheidbarkeit \hfill{} \normalfont{(4 Punkte)}}

% lstlisting cannot go inside fbox, so we nee to do this shit
\newsavebox{\codebox}
\begin{lrbox}{\codebox}
   \begin{lstlisting}[escapeinside={*}{*}]
   for n = 1,...,*$\infty$*:
      foreach *$\vec{i}$* in *$\{1,...,n\}^k$*:
         if *$\op\limits_{i=1}^na_{\vec{i}[i]} - \op\limits_{i=1}^nb_{\vec{i}[i]}$ = $\op\limits_{i=1}^nc_{\vec{i}[i]}$*:
            return true
   \end{lstlisting}
\end{lrbox}

\noindent\fcolorbox{lightgray}{lightgray!20}{
   \begin{minipage}[t][9cm]{.95\textwidth}
      Wir beweisen die Semi-Entscheidbarkeit, indem wir einen Algorithmus
      angeben, der eine Lösung findet, sofern sie existiert.

      \begin{center}
         \usebox{\codebox}
      \end{center}

      Es gibt in einer $n$-Elementigen Menge nur endlich viele Auswahlen von $k$
      Elementen. Gibt es eine Lösung, wird sie also irgendwann gefunden.
   \end{minipage}
}

\subsection*{(c) Unentscheidbarkeit \hfill{} \normalfont{(4 Punkte)}}

Beschreiben Sie kurz die notwendige Reduktion (von? nach? wie?) um zu begründen,
warum das Problem nicht entscheidbar ist:

\vspace{4pt}
\noindent\fcolorbox{lightgray}{lightgray!20}{
   \begin{minipage}[t][9cm]{.95\textwidth}
      Es muss ein unentscheidbares Problem auf das \emph{Klebeproblem} reduziert
      werden. Es bietet sich hierfür das PCP an. Gegeben eine PCP-Instanz mit
      Tupelmenge 
      \begin{equation*}
         \mathcal{M} = \{(x_i,y_i) \mid 1\le i\le k, x,y\in\mathbb{N}\}
      \end{equation*} und $|\mathcal{M}| = k$, erstellen wir eine Klebeinstanz
      mit 
      \begin{equation*}
         \mathcal{A} := \{x_i \mid 1\le i\le k,
         (x_i,y_i)\in\mathcal{M}\},~\mathcal{B} := \{0\}^k,~\mathcal{C} := \{y_i \mid
         1\le i \le k, (x_i,y_i)\in\mathcal{M}\}.
      \end{equation*} 
      Offensichtlich ist die PCP-Instanz lösbar, genau
      dann, wenn die Klebeinstanz lösbar ist, da
      \begin{equation*}
         \op\limits_{i=1}^ka_{s(i)} - \{0\}^k = \op\limits_{i=1}^kc_{s(i)}
      \end{equation*}
      gleichbedeutend ist mit der Aussage, dass links dieselbe Zahl steht wie
      rechts. Da links die $x_i$ der PCP-Instanz stehen und rechts die $y_i$,
      sind die Problemstellungen identisch. Man muss eine Auswahl an Indizes
      finden (wobei nicht alle $k$ vorkommen müssen, und Indizes mehrfach
      verwended werden können), sodass die konkatenierten $x_i$ gleich den
      konkatenierten $y_i$ sind, was genau der Klebeoperation entspricht.
   \end{minipage}
}
\section*{Aufgabe 9: {\sffamily \textbf{\itshape P}} vs. {\sffamily \textbf{\itshape NP}} \hfill{}(20 Punkte)}

\subsection*{(a) Definition \hfill{} \normalfont{(4 Punkte)}}

Definieren Sie die Komplexitätsklasse \textbf{\sffamily\itshape{P}}.
\end{document}
